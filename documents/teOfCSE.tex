\documentclass[12pt]{scrartcl}

\usepackage[printwatermark,disablegeometry]{xwatermark}

\usepackage{epsfig,amssymb}

\usepackage{xcolor}
\usepackage{graphicx}
\usepackage{epstopdf}
\usepackage{multirow}

\definecolor{darkred}{rgb}{0.5,0,0}
\definecolor{darkgreen}{rgb}{0,0.5,0}
\usepackage{hyperref}
\hypersetup{
  letterpaper,
  colorlinks,
  linkcolor=red,
  citecolor=darkgreen,
  menucolor=darkred,
  urlcolor=blue,
  bookmarks=true,
  pdfpagemode=none,
  pdftitle={Te of CSE Office Hours},
  pdflang={en},
  pdfauthor={Christopher M. Bourke},
  pdfcreator={$ $Id: cv-us.tex,v 1.28 2009/01/01 00:00:00 cbourke Exp $ $},
  pdfsubject={PhD Thesis},
  pdfkeywords={}
}

\usepackage{fullpage}
\usepackage{tikz}
\pagestyle{empty} %
\usepackage{subfigure}

\definecolor{MyDarkBlue}{rgb}{0,0.08,0.45}
\definecolor{MyDarkRed}{rgb}{0.45,0.08,0}
\definecolor{MyDarkGreen}{rgb}{0.08,0.45,0.08}

\definecolor{mintedBackground}{rgb}{0.95,0.95,0.95}
\definecolor{mintedInlineBackground}{rgb}{.90,.90,1}

\newwatermark[allpages=true,scale=5,textmark=Draft]{}

\usepackage[newfloat=true]{minted}

\setminted{mathescape,
           linenos,
           autogobble,
           frame=none,
           framesep=2mm,
           framerule=0.4pt,
           %label=foo,
           xleftmargin=2em,
           xrightmargin=0em,
           %startinline=true,  %PHP only, allow it to omit the PHP Tags *** with this option, variables using dollar sign in comments are treated as latex math
           numbersep=10pt, %gap between line numbers and start of line
           style=default} %syntax highlighting style, default is "default"

\setmintedinline{bgcolor={mintedBackground}}
%doesn't work with the above workaround:
\setminted{bgcolor={mintedBackground}}
\setminted[text]{bgcolor={mintedBackground},linenos=false,autogobble,xleftmargin=1em}
%\setminted[php]{bgcolor=mintedBackgroundPHP} %startinline=True}
\SetupFloatingEnvironment{listing}{name=Code Sample}
\SetupFloatingEnvironment{listing}{listname=List of Code Samples}

\setlength{\parindent}{0pt} %
\setlength{\parskip}{.25cm}
\newcommand{\comment}[1]{}

\usepackage{amsmath}
\usepackage{algorithm2e}
\SetKwInOut{Input}{input}
\SetKwInOut{Output}{output}
%NOTE: you can embed algorithms in solutions, but they cannot be floating objects; use [H] to make them non-floats

\usepackage{lastpage}

%\usepackage{titling}
\usepackage{fancyhdr}
\renewcommand*{\titlepagestyle}{fancy}
\pagestyle{fancy}
%\renewcommand*{\titlepagestyle}{fancy}
%\fancyhf{}
%\rhead{Computer Science I}
%\lhead{Guides and tutorials}
\renewcommand{\headrulewidth}{0.0pt}
\renewcommand{\footrulewidth}{0.4pt}
\lfoot{\Title\ -- Computer Science I}
\cfoot{~}
\rfoot{\thepage\ / \pageref*{LastPage}}

\makeatletter
\title{Te of CSE Office Hours}\let\Title\@title
\subtitle{Computer Science I\\
{\small
\vskip.5cm
Department of Computer Science \& Engineering \\
University of Nebraska--Lincoln}
\vskip-2cm}
%\author{Dr.\ Chris Bourke}
\date{~}
\makeatother

\begin{document}

\maketitle

\hrule

\section*{DRAFT}

\begin{itemize}
  \item Have your IDE or coding environment open, setup and ready to go before you ask for help.
  \item You must make a substantial and honest attempt to solve your problem before you ask for help.
  \item Look for similar problems or examples from the lectures, labs, other assignments, text book, etc.
  \item Look to your peers for help!  To the level that you are allowed to collaborate, ask someone around you.  On the same note, if someone around you asks for help, lend it!
  \item Be specific with your problem: be able to fully describe what the issue is and the steps you've taken to resolve it so far and what you think the problem might be.
  \item Show Respect.  This includes respecting the time of the Learning Assistants, respecting your instructor and respecting your fellow students.
\end{itemize}

Student says, I don't know where to start
\begin{itemize}
  \item Have you read the handout
  \item Have you walked through any examples provided on a separate sheet of paper/whiteboard
  \item Have you worked through another example of your own creation?
  \item Have you outlined a rough solution (via comments)
  \item Have you "Free written" major components of the program that you'll need to write
\end{itemize}

Student has a bug or problem they don't know how to solve
\begin{itemize}
  \item Have them describe the problem
  \item Have them describe the solutions they have attempted so far
  \item Can they reproduce the problem consistently?
  \item Have them show you their test case(s)
  \item Have them identify the line(s) it is occurring on
  \item Have them run a debugger up to the point of failure and explore with them
\end{itemize}



  


\end{document}
