\documentclass[12pt]{scrartcl}


\usepackage{epsfig,amssymb}

\usepackage{xcolor}
\usepackage{graphicx}
\usepackage{epstopdf}
\usepackage{multirow}

\definecolor{darkred}{rgb}{0.5,0,0}
\definecolor{darkgreen}{rgb}{0,0.5,0}
\usepackage{hyperref}
\hypersetup{
  letterpaper,
  colorlinks,
  linkcolor=red,
  citecolor=darkgreen,
  menucolor=darkred,
  urlcolor=blue,
  pdfpagemode=none,
}

\usepackage{fullpage}
\usepackage{tikz}
\pagestyle{empty} %
\usepackage{subfigure}

\definecolor{MyDarkBlue}{rgb}{0,0.08,0.45}
\definecolor{MyDarkRed}{rgb}{0.45,0.08,0}
\definecolor{MyDarkGreen}{rgb}{0.08,0.45,0.08}

\definecolor{mintedBackground}{rgb}{0.95,0.95,0.95}
\definecolor{mintedInlineBackground}{rgb}{.90,.90,1}

\usepackage[newfloat=true]{minted}

\setminted{mathescape,
           linenos,
           autogobble,
           frame=none,
           framesep=2mm,
           framerule=0.4pt,
           %label=foo,
           xleftmargin=2em,
           xrightmargin=0em,
           %startinline=true,  %PHP only, allow it to omit the PHP Tags *** with this option, variables using dollar sign in comments are treated as latex math
           numbersep=10pt, %gap between line numbers and start of line
           style=default} %syntax highlighting style, default is "default"

\setmintedinline{bgcolor={mintedBackground}}
%doesn't work with the above workaround:
\setminted{bgcolor={mintedBackground}}
\setminted[text]{bgcolor={mintedBackground},linenos=false,autogobble,xleftmargin=1em}
%\setminted[php]{bgcolor=mintedBackgroundPHP} %startinline=True}
\SetupFloatingEnvironment{listing}{name=Code Sample}
\SetupFloatingEnvironment{listing}{listname=List of Code Samples}

\setlength{\parindent}{0pt} %
\setlength{\parskip}{.25cm}
\newcommand{\comment}[1]{}

\usepackage{amsmath}
\usepackage{algorithm2e}
\SetKwInOut{Input}{input}
\SetKwInOut{Output}{output}
%NOTE: you can embed algorithms in solutions, but they cannot be floating objects; use [H] to make them non-floats

\usepackage{lastpage}

%\usepackage{titling}
\usepackage{fancyhdr}
\renewcommand*{\titlepagestyle}{fancy}
\pagestyle{fancy}
%\renewcommand*{\titlepagestyle}{fancy}
%\fancyhf{}
%\rhead{Computer Science I}
%\lhead{Guides and tutorials}
\renewcommand{\headrulewidth}{0.0pt}
\renewcommand{\footrulewidth}{0.4pt}

\lhead{~}
\chead{~}
\rhead{~}
\lfoot{\Title}
\cfoot{~}
\rfoot{\thepage\ / \pageref*{LastPage}}

\makeatletter
\title{Code Troubleshooting Guide}\let\Title\@title
\subtitle{Computer Science I\\
{\small
\vskip1cm
Department of Computer Science \& Engineering \\
University of Nebraska--Lincoln}
\vskip-1cm}
%\author{Dr.\ Chris Bourke}
\date{~}
\makeatother

\begin{document}

\maketitle

\hrule

\section*{Troubleshooting Buggy Code}

Often you'll encounter a bug or problem with your code that 
leads to unexpected results, segmentation faults or other issues.  
Learning how to approach these problems with a rigorous and
systematic approach is an essential skill in software development.
Though there are many approaches, we'll list a few basic steps
that you should try before throwing in the towel.

\begin{enumerate}
  \item Test!  Write test cases and test suites or use a formal
  unit testing framework.  Writing and using unit tests and developing
  test cases Test cases helps to provide assurance that your
  software is correct.  The more formal the testing, the higher assurance
  you have and the more maintainable and repeatable your tests are.
  It also has additional benefits: by writing tests it forces you to
  think/rethink your software design, understand the problem better, and
  more rigorously outlines your expectations on how the software should
  work.
  
  \item Use a linter.  Always use the \mintinline{text}{-Wall} flag 
  when compiling.  This makes the compiler reports \emph{all} warnings.
  These are not necessarily problems with your program, but may (and often
  do) indicate code that can lead to bugs or are bad practices.  

  \item Use a static analyzer.  Similar to a linter, there are tools that
  you can use that perform \emph{static analysis} of code.  This goes
  beyond merely checking the syntax or code for common errors and performs
  sophisticated analysis of code to find potential bugs and issues.  Two
  static analysis tools are available on CSE:
  \begin{itemize}
    \item Clang is another compiler (actually a front-end for the LLVM 
    compiler) and has a similar linter/analyzer capability.  It often
    catches the same issues as GCC's linter, but often catches more
    or different issues.  You can use it as follows:
    
    \mintinline{text}{clang --analyze foo.c}
    
    \item Clang also has a Static Analyzer tool (see 
    \url{http://clang-analyzer.llvm.org/}, 
    \url{http://clang-analyzer.llvm.org/available_checks.html})
    that you can use in your compilation or build process.  
    Examples:
    
    Using gcc: \\
    \mintinline{text}{scan-build gcc foo.c}
    
    Using clang: \\
    \mintinline{text}{scan-build clang foo.c}

    Using make: \\
    \mintinline{text}{scan-build make}

    \item Facebook's Infer (\url{https://github.com/facebook/infer}) 
    is another static analysis tool that can be used on C, Java, and 
    several other programming languages.  Full documentation can be
    found here: \url{https://fbinfer.com/docs/infer-workflow.html}, but
    we've installed it on CSE and you can use it in a two phase process.
    
    First, capture static information using infer when you compile your
    program: 
     
    \mintinline{text}{infer capture -- gcc hello.c}

    This produces analysis data in a folder named \mintinline{text}{infer-out}
    which you can then run static analysis on: 

    \mintinline{text}{infer analyze}


  \end{itemize}
  
  \item Use a dynamic analyzer.  Valgrind is a dynamic analysis tool 
  that can monitor a program and analyze the resources it uses and
  operations it performs.  Dynamic analysis is different in that it
  performs its analysis by running the program instead of just looking
  at the source code.  To use it, you first need to use the 
  \mintinline{text}{-g} flag to compile in order to preserve function
  names and line numbers in the analysis.  Once compiled, you can start
  Valgrind using
  
  \mintinline{text}{valgrind --leak-check=full --show-leak-kinds=all ./a.out}

  It will run your program and produce a report of any issues including
  potential (and ``definite'') memory leaks.

  \item Use a debugger.  GNU's Debugger (GDB) is a powerful tool that
  allows you to run, test, and debug a program by simulating its execution.
  You can pause and resume the program, examine variable values, and
  step through the program line-by-line.  A full introduction to GDB
  is available in the course resources:
  \begin{itemize}
    \item Video Introduction to GDB: \url{https://www.youtube.com/watch?v=bWH-nL7v5F4&index=43}
    \item Hack 9.0: \url{https://github.com/cbourke/CSCE155-Hack9.0}
  \end{itemize}
  
  
  \item ``Perf'' Test.  Performance testing may be necessary if you have 
  code that compiles, runs, and may even work, but seems to be inefficient.
  You can use a \emph{profiler} to monitor the execution of a program
  and determine which function(s) or sections of code take the most time
  or resources.  GNU's Gprof performance analysis tool can be used to
  profile a program.  First, compile using the \mintinline{text}{-pg} flag:
  
  \mintinline{text}{gcc -pg foo.c}
  
  to preserve identifiers and make it so that gprof can profile the 
  program.  Then run the program normally.  After it has concluded 
  executing, it will have created a \mintinline{text}{gmon.out} file
  which contains a gprof report.  To view it you can use the following:
  
  \mintinline{text}{gprof a.out}
  
  
\end{enumerate}


\end{document}
