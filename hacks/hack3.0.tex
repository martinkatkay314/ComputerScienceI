\documentclass[12pt]{scrartcl}


\usepackage{epsfig,amssymb}

\usepackage{xcolor}
\usepackage{graphicx}
\usepackage{epstopdf}
\usepackage{multirow}
\usepackage{float}

\definecolor{darkred}{rgb}{0.5,0,0}
\definecolor{darkgreen}{rgb}{0,0.5,0}
\usepackage[pdfusetitle]{hyperref}
\hypersetup{
  letterpaper,
  colorlinks,
  linkcolor=red,
  citecolor=darkgreen,
  menucolor=darkred,
  urlcolor=blue,
  pdfpagemode=none,
}

\usepackage{fullpage}
\usepackage{tikz}
\pagestyle{empty} %
%obsolete: \usepackage{subfigure}
%use: 
\usepackage{subcaption}

\definecolor{MyDarkBlue}{rgb}{0,0.08,0.45}
\definecolor{MyDarkRed}{rgb}{0.45,0.08,0}
\definecolor{MyDarkGreen}{rgb}{0.08,0.45,0.08}

\definecolor{mintedBackground}{rgb}{0.95,0.95,0.95}
\definecolor{mintedInlineBackground}{rgb}{.90,.90,1}

\usepackage[newfloat=true]{minted}

\setminted{mathescape,
           linenos,
           autogobble,
           frame=none,
           framesep=2mm,
           framerule=0.4pt,
           %label=foo,
           xleftmargin=2em,
           xrightmargin=0em,
           %startinline=true,  %PHP only, allow it to omit the PHP Tags *** with this option, variables using dollar sign in comments are treated as latex math
           numbersep=10pt, %gap between line numbers and start of line
           style=default} %syntax highlighting style, default is "default"

\setmintedinline{bgcolor={mintedBackground}}
%doesn't work with the above workaround:
\setminted{bgcolor={mintedBackground}}
\setminted[text]{bgcolor={mintedBackground},linenos=false,autogobble,xleftmargin=1em}
%\setminted[php]{bgcolor=mintedBackgroundPHP} %startinline=True}
\SetupFloatingEnvironment{listing}{name=Code Sample}
\SetupFloatingEnvironment{listing}{listname=List of Code Samples}

\setlength{\parindent}{0pt} %
\setlength{\parskip}{.25cm}
\newcommand{\comment}[1]{}

\usepackage{amsmath}
\usepackage{algorithm2e}
\SetKwInOut{Input}{input}
\SetKwInOut{Output}{output}
%NOTE: you can embed algorithms in solutions, but they cannot be floating objects; use [H] to make them non-floats

\usepackage{lastpage}

%\usepackage{titling}
\usepackage{fancyhdr}
\renewcommand*{\titlepagestyle}{fancy}
\pagestyle{fancy}
%\fancyhf{}
%\rhead{Computer Science I}
%\lhead{Guides and tutorials}
\renewcommand{\headrulewidth}{0.0pt}
\renewcommand{\footrulewidth}{0.4pt}
\lfoot{\Title\ -- Computer Science I}
\cfoot{~}
\rfoot{\thepage\ / \pageref*{LastPage}}


\makeatletter
\title{Hack 3.0}\let\Title\@title
\subtitle{Computer Science I\\
{\small
\vskip1cm
Department of Computer Science \& Engineering \\
University of Nebraska--Lincoln}
\vskip-1cm}
%\author{Dr.\ Chris Bourke}
\date{~}
\makeatother

\begin{document}

\maketitle

\hrule

\section*{Introduction}

Hack session activities are small weekly programming assignments intended
to get you started on full programming assignments.  Collaboration is allowed
and, in fact, \emph{highly encouraged}.  You may start on the activity before
your hack session, but during the hack session you must either be actively 
working on this activity or \emph{helping others} work on the activity.
You are graded using the same rubric as assignments so documentation, style, 
design and correctness are all important.

%\subsection*{Rubric}
%\begin{table}[H]
%\begin{tabular}{ll}
%Category       & Point Value \\
%Style          & 2           \\
%Documentation  & 2           \\
%Design         & 5           \\
%Correctness    & 16          \\
%\textbf{Total} & \textbf{25}
%\end{tabular}
%\end{table}

 

Correctness: Formatting variations are perfectly fine.  You need to
report *just as much* information as the expected output.
Points are awarded proportionally per test case.

\section*{Problem Statement}

A cell phone broadband provider sells monthly data plans to its
customers.  Each data plan is for 30 days and each plan gives
customers a certain number of Gigabytes (GB) which must be used 
each month or they are lost (no ``rollover'').  The provider wants
to help customers understand if they are using their monthly data
too quickly or if they can afford to use more.

In this exercise, you will write an application for the company
to help customers track their mobile data usage.  Write a program
that reads the following pieces of data as \emph{command line arguments}.
\begin{itemize}
  \item Number of GB in the plan per 30 day period
  \item The current day in the 30 day period (in the range 1, first day, 
  30 for the last day)
  \item The total number of GB used so far
\end{itemize}
The program should then compute whether the customer is over, 
under, or right on the \emph{average daily usage} of their plan.  
It should also inform them of how many GB are left and how many, 
on average, they can use per day for the rest of the 30 day period.  
Of course, if they've run out of data, it should inform them of that 
too.

For example, if the user enters $15, 10, 13$ for each piece of data
respectively, your program should print out something similar to 
the following.

\begin{minted}{text}
10 days used, 20 days remaining
Average daily use: 1.3 GB/day

You are EXCEEDING your average daily use (0.50 GB/day).
Continuing this high usage, you'll exceed your data plan by
24 GB.

To stay below your data plan, use no more than 0.1 GB/day.
\end{minted}

If the user is under their average daily use, a different
message should be presented.  

\section*{Instructions}

\begin{itemize}
  \item You are encouraged to collaborate any number of students 
  before, during, and after your scheduled hack session.  
  \item Design at least 3 test cases \emph{before} you begin
  designing or implementing your program.  Test cases are 
  input-output pairs that are known to be correct using means
  other than your program.
  \item Include the name(s) of everyone who worked together on
  this activity in your source file's header.
  \item Name your program \mintinline{text}{dataPlan.c}, and
  turn it in via webhandin, making sure that it runs and executes
  correctly in the webgrader.  Each individual student will need
  to hand in their own copy and will receive their own individual
  grade.
\end{itemize}
  


\end{document}
