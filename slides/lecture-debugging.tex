%%%%%%%%%%%%%%%%%%%%%%%%%%%%%%%%%%%%%%%%%%%%%%%%%%%%%
%
%  Template
%  Beamer Presentation by Chris Bourke
%
%%%%%%%%%%%%%%%%%%%%%%%%%%%%%%%%%%%%%%%%%%%%%%%%%%%%%%%%%%%%%%%%%%%%%%%

\documentclass[]{beamer}
%\documentclass[handout]{beamer}

\geometry{papersize={16cm,9cm}}

% For handout version:
%\usetheme[hideothersubsections,slidenumbers]{UNLTheme}
\usetheme[hideothersubsections]{UNLTheme}
\usepackage{amssymb}
\input{StandardCommands}
\usepackage[linesnumbered,ruled,vlined]{algorithm2e}
\SetKwComment{Comment}{//}{}
\DontPrintSemicolon
\SetKwSty{textsc} %
%\SetAlFnt{\scriptsize} %
\SetKwInOut{Input}{Input} %
\SetKwInOut{Output}{Output} %
%\setalcapskip{1em} % changed to
\SetAlCapSkip{1em}
\setlength{\algomargin}{2em} %
%\Setvlineskip{0em} % changed to:
\SetVlineSkip{0em}

\usepackage{tikz}
\usetikzlibrary{fadings}
\usetikzlibrary{shapes.geometric,shapes.symbols}
\usetikzlibrary{calc,shapes.multipart,chains,arrows}
\usetikzlibrary{arrows.meta,calc,shapes.multipart,chains,arrows}
%\usetikzlibrary{calc,shapes.multipart,chains,arrows}
%%\usetikzlibrary{backgrounds}
\usetikzlibrary{backgrounds}
\usetikzlibrary{decorations.pathreplacing}
\usetikzlibrary{decorations.pathmorphing}
\tikzset{onslide/.code args={<#1>#2}{%
  \only<#1>{\pgfkeysalso{#2}} % \pgfkeysalso doesn't change the path
}}
\tikzset{temporal/.code args={<#1>#2#3#4}{%
  \temporal<#1>{\pgfkeysalso{#2}}{\pgfkeysalso{#3}}{\pgfkeysalso{#4}} % \pgfkeysalso doesn't change the path
}}


\tikzset{
    fading speed/.code={
        \pgfmathtruncatemacro\tikz@startshading{50-(100-#1)*0.25}
        \pgfmathtruncatemacro\tikz@endshading{50+(100-#1)*0.25}
        \pgfdeclareverticalshading[%
            tikz@axis@top,tikz@axis@middle,tikz@axis@bottom%
        ]{axis#1}{100bp}{%
            color(0bp)=(tikz@axis@bottom);
            color(\tikz@startshading)=(tikz@axis@bottom);
            color(50bp)=(tikz@axis@middle);
            color(\tikz@endshading)=(tikz@axis@top);
            color(100bp)=(tikz@axis@top)
        }
        \tikzset{shading=axis#1}
    }
}

\usepackage{multirow}
\usepackage{multicol}

\definecolor{steelblue}{rgb}{0.2745,0.5098,0.7059}
\definecolor{limegreen}{RGB}{50,205,50}
\hypersetup{
    colorlinks = true,
    urlcolor = {steelblue},
    linkbordercolor = {white}
}

\definecolor{mintedBackground}{rgb}{0.95,0.95,0.95}
\definecolor{mintedInlineBackground}{rgb}{.90,.90,1}

%\usepackage{newfloat}
\usepackage{minted}
\setminted{mathescape,
               linenos,
               autogobble,
               frame=none,
               fontsize=\small,
               framesep=2mm,
               framerule=0.4pt,
               %label=foo,
               xleftmargin=2em,
               xrightmargin=0em,
               startinline=true,  %PHP only, allow it to omit the PHP Tags *** with this option, variables using dollar sign in comments are treated as latex math
               numbersep=10pt, %gap between line numbers and start of line
               style=default, %syntax highlighting style, default is "default"
               			    %gallery: http://help.farbox.com/pygments.html
			    	    %list available: pygmentize -L styles
               bgcolor=mintedBackground} %prevents breaking across pages
               
\setmintedinline{bgcolor={mintedInlineBackground}}
\setminted[text]{bgcolor={},linenos=false,autogobble,xleftmargin=1em}

\tikzstyle{decision} = [diamond, draw, fill=yellow!20, 
    text width=6em, text badly centered, node distance=5cm, inner sep=0pt]
\tikzstyle{block} = [rectangle, draw, fill=blue!20, 
    text width=5em, text centered, node distance=5cm, minimum height=4em]
\tikzstyle{action} = [rectangle, draw, fill=green!20, 
    text width=5em, text centered, rounded corners, node distance=5cm, minimum height=4em]
\tikzstyle{line} = [draw, -latex']

\title[~]{Computer Science I}
\subtitle{Debugging}
\author[~]{Dr.\ Chris Bourke\\ \email{cbourke@cse.unl.edu}} %
\date{~}

\begin{document}

\begin{frame}
  \titlepage
\end{frame}

\setbeamertemplate{section in toc}{\inserttocsectionnumber.~\inserttocsection}
\setbeamercolor{section in toc}{fg=black}
%\setbeamertemplate{subsection in toc}{~} %\inserttocsubsection\\}

\begin{frame}
  \frametitle{Outline}
%{\footnotesize
%\begin{NoHyper}
%  \tableofcontents[hideallsubsections]
%\end{NoHyper}
%}

\setbeamercolor{enumerate item}{bg=white,fg=black}
\setbeamercolor{item}{bg=white,fg=black}
\setbeamercolor{item projected}{bg=white,fg=black}
\setbeamercolor{enumerate subitem}{fg=red!80!black}
\setbeamertemplate{enumerate items}[default]
\begin{enumerate}
  \item Introduction 
  \item Demonstration
\end{enumerate}

\end{frame}

\section{Introduction}

\begin{frame}
    \frametitle{}
    \framesubtitle{}
    
    \begin{center}
    {\Huge Part I: Introduction}\\
    {\Large ~}
    \end{center}

\end{frame}


\begin{frame}[fragile]
    \frametitle{Debugging}
    \framesubtitle{}

\begin{itemize}[<+->]
  \item A \emph{bug} is an error or flaw in a program or system that
  produces incorrect or unexpected results.
  \item Process of \emph{debugging} involves \emph{identifying} and
  \emph{resolving} bugs so that the program/system performs as expected
  \item General techniques \& strategies
  \item Demonstration of a \emph{debugger}
\end{itemize}

\end{frame}

\begin{frame}[fragile]
    \frametitle{Types of Errors}
    \framesubtitle{}

\begin{itemize}[<+->]
  \item Syntax Errors
  \item Runtime Errors
  \item Logic Errors
\end{itemize}
    
\end{frame}

\begin{frame}[fragile]
    \frametitle{Poor Man's Debugging}
    \framesubtitle{}

\begin{itemize}[<+->]
  \item Up to now: \emph{poor man's debuggin}
  \item Using print statements to determine value(s) of variables at different parts of a program
  \item Completely insufficient in practice
  \item Lots of manual time and effort
  %you have to write and remove a lot of code
  \item Does not create reproducible tests/artifacts
  \item Extremely fragile and error prone
  %standard output is not deterministic; problems can   
  \item Ad-hoc testing can serve a purpose but is similarly insufficient
  \item Even worse: blind coding, aimlessly changing things with not thought
% worst form: stabs in the dark, aimlessly changing things with no thought; "trying anything"
\end{itemize}

\end{frame}

\begin{frame}[fragile]
  \frametitle{Debugging \& Testing}
  \framesubtitle{}

\begin{itemize}[<+->]
  \item A bug indicates not only an error in your program but also
  \item that your \emph{tests are insufficient}
  \item Unit tests \& good code coverage should \emph{prevent} bugs 
  \item Debugging involves not only correcting the issue but also
  \item correcting the test suite by adding tests to cover the identified bug
\end{itemize}

\onslide<6->{
\begin{quote}
Learning and practicing proper testing is a result of accepting that you will not write perfect code.
\end{quote}
}
\onslide<7->{
\begin{quote}
Learning and practicing proper debugging is a result of accepting that you will not write perfect tests.
\end{quote}
}

\end{frame}

\begin{frame}
  \frametitle{Debugging Strategies}
  \framesubtitle{}

General Debugging Strategies
\begin{itemize}[<+->] 
  \item Understand the bug: 
  \begin{itemize}
    \item \emph{How} is the program failing?  
    \item \emph{What} inputs or conditions are causing the error
    \item Is it a bug in our code or an error in our \emph{understanding}
    %documentation is wrong or our expectations of the library we're using are wrong
  \end{itemize}
  \item Reproduce the bug:
  \begin{itemize}
    \item Forces us to formally identify \emph{correct behavior}
    \item Design a (reusable) test case that highlights the bug
    \item Gives us something to work with
  \end{itemize}
\end{itemize}

\end{frame}

\begin{frame}
  \frametitle{Debugging Strategies}
  \framesubtitle{}

\begin{itemize}[<+->] 
    \item Isolate the bug
  \begin{itemize}
    \item Rule out other issues: configuration, out-of-synch code, etc.
    \item Reexamine assumptions \& understanding 
    \item Narrow the failure point down to a testable unit (module/function/line)
  \end{itemize}
  \item Investigate
  \begin{itemize}
    \item Use your knowledge of the code to formulate a hypothesis about what is wrong
    \item Test your hypothesis
    \item Use a debugging tool to walk through your code
  \end{itemize}
  \item Fix it, test it (and regression test it), document it
\end{itemize}
  
\end{frame}

\begin{frame}
    \frametitle{}
    \framesubtitle{}
    
    \begin{center}
    {\Huge Part II: Demonstration}\\
    {\Large ~}
    \end{center}

\end{frame}


\begin{frame}[fragile]
    \frametitle{Debuggers}
    \framesubtitle{}

\begin{itemize}[<+->]
  \item A \emph{debugger} is a program that simulates/runs 
  another program and allows you to:
  \begin{itemize}
    \item Pause and continue its execution
    \item Set ``break points'' or conditions where the execution pauses so you can look at its state
    \item View and ``watch'' variable values
    \item Step through the program line-by-line (or instruction by instruction)
  \end{itemize}
  \item GNU Debugger (gdb)
  %GUI's available, but we'll go with command line 
\end{itemize}

\end{frame}


\begin{frame}[fragile]
    \frametitle{GDB: Getting Started}
    \framesubtitle{}

\begin{itemize}[<+->]
  \item Compile for debugging:\\
  \mintinline{text}{gcc -Wall -g program.c}
  \item Preserves identifiers and symbols
  \item Start GDB:\\
  \mintinline{text}{gdb a.out}
  \item Optionally start with CLAs: \\
  \mintinline{text}{gdb --args a.out arg1 arg2}
  \item Can also set in GDB
\end{itemize}

\end{frame}


\begin{frame}[fragile]
    \frametitle{Useful GDB Commands}
    \framesubtitle{}

\begin{itemize}
  \item Refresh the display: \mintinline{text}{refresh} (or control-L)
  \item Run your program: \mintinline{text}{run}
  \item See your code: \mintinline{text}{layout next}
  \item Set a break point: \mintinline{text}{break POINT}, can be a line number, function name, etc.
  \item Step: \mintinline{text}{next} (\mintinline{text}{n} for short)
  \item Continue (to next break point): \mintinline{text}{continue}
  \item Print a variable's value: \mintinline{text}{print VARIABLE}
  \item Print an array: \mintinline{text}{print *arr@len}
  \item Watch a variable for changes: \mintinline{text}{watch VARIABLE}
\end{itemize}  

\end{frame}


\begin{frame}[fragile]
    \frametitle{Demonstration}
    \framesubtitle{}

Demonstration    
    
\end{frame}

\end{document} 
