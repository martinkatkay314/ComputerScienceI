%%%%%%%%%%%%%%%%%%%%%%%%%%%%%%%%%%%%%%%%%%%%%%%%%%%%%
%
%  Template
%  Beamer Presentation by Chris Bourke
%
%%%%%%%%%%%%%%%%%%%%%%%%%%%%%%%%%%%%%%%%%%%%%%%%%%%%%%%%%%%%%%%%%%%%%%%

\documentclass[]{beamer}
%\documentclass[handout]{beamer}

\geometry{papersize={16cm,9cm}}

% For handout version:
%\usetheme[hideothersubsections,slidenumbers]{UNLTheme}
\usetheme[hideothersubsections]{UNLTheme}
\usepackage{amssymb}
\input{StandardCommands}
\usepackage[linesnumbered,ruled,vlined]{algorithm2e}
\SetKwComment{Comment}{//}{}
\DontPrintSemicolon
\SetKwSty{textsc} %
%\SetAlFnt{\scriptsize} %
\SetKwInOut{Input}{Input} %
\SetKwInOut{Output}{Output} %
%\setalcapskip{1em} % changed to
\SetAlCapSkip{1em}
\setlength{\algomargin}{2em} %
%\Setvlineskip{0em} % changed to:
\SetVlineSkip{0em}

\usepackage{tikz}
\usetikzlibrary{fadings}
\usetikzlibrary{shapes.geometric,shapes.symbols}
\usetikzlibrary{calc,shapes.multipart,chains,arrows}
\usetikzlibrary{arrows.meta,calc,shapes.multipart,chains,arrows}
%\usetikzlibrary{calc,shapes.multipart,chains,arrows}
%%\usetikzlibrary{backgrounds}
\usetikzlibrary{backgrounds}
\usetikzlibrary{decorations.pathreplacing}
\usetikzlibrary{decorations.pathmorphing}
\tikzset{onslide/.code args={<#1>#2}{%
  \only<#1>{\pgfkeysalso{#2}} % \pgfkeysalso doesn't change the path
}}
\tikzset{temporal/.code args={<#1>#2#3#4}{%
  \temporal<#1>{\pgfkeysalso{#2}}{\pgfkeysalso{#3}}{\pgfkeysalso{#4}} % \pgfkeysalso doesn't change the path
}}


\tikzset{
    fading speed/.code={
        \pgfmathtruncatemacro\tikz@startshading{50-(100-#1)*0.25}
        \pgfmathtruncatemacro\tikz@endshading{50+(100-#1)*0.25}
        \pgfdeclareverticalshading[%
            tikz@axis@top,tikz@axis@middle,tikz@axis@bottom%
        ]{axis#1}{100bp}{%
            color(0bp)=(tikz@axis@bottom);
            color(\tikz@startshading)=(tikz@axis@bottom);
            color(50bp)=(tikz@axis@middle);
            color(\tikz@endshading)=(tikz@axis@top);
            color(100bp)=(tikz@axis@top)
        }
        \tikzset{shading=axis#1}
    }
}

\usepackage{multirow}
\usepackage{multicol}

\definecolor{steelblue}{rgb}{0.2745,0.5098,0.7059}
\definecolor{limegreen}{RGB}{50,205,50}
\hypersetup{
    colorlinks = true,
    urlcolor = {steelblue},
    linkbordercolor = {white}
}

\definecolor{mintedBackground}{rgb}{0.95,0.95,0.95}
\definecolor{mintedInlineBackground}{rgb}{.90,.90,1}

%\usepackage{newfloat}
\usepackage{minted}
\setminted{mathescape,
               linenos,
               autogobble,
               frame=none,
               fontsize=\small,
               framesep=2mm,
               framerule=0.4pt,
               %label=foo,
               xleftmargin=2em,
               xrightmargin=0em,
               startinline=true,  %PHP only, allow it to omit the PHP Tags *** with this option, variables using dollar sign in comments are treated as latex math
               numbersep=10pt, %gap between line numbers and start of line
               style=default, %syntax highlighting style, default is "default"
               			    %gallery: http://help.farbox.com/pygments.html
			    	    %list available: pygmentize -L styles
               bgcolor=mintedBackground} %prevents breaking across pages
               
\setmintedinline{bgcolor={mintedInlineBackground}}
\setminted[text]{bgcolor={},linenos=false,autogobble,xleftmargin=1em}

\tikzstyle{decision} = [diamond, draw, fill=yellow!20, 
    text width=6em, text badly centered, node distance=5cm, inner sep=0pt]
\tikzstyle{block} = [rectangle, draw, fill=blue!20, 
    text width=5em, text centered, node distance=5cm, minimum height=4em]
\tikzstyle{action} = [rectangle, draw, fill=green!20, 
    text width=5em, text centered, rounded corners, node distance=5cm, minimum height=4em]
\tikzstyle{line} = [draw, -latex']

\title[~]{Computer Science I}
\subtitle{Strings}
\author[~]{Dr.\ Chris Bourke\\ \email{cbourke@cse.unl.edu}} %
\date{~}

\begin{document}

\begin{frame}
  \titlepage
\end{frame}

\setbeamertemplate{section in toc}{\inserttocsectionnumber.~\inserttocsection}
\setbeamercolor{section in toc}{fg=black}
%\setbeamertemplate{subsection in toc}{~} %\inserttocsubsection\\}

\begin{frame}
  \frametitle{Outline}
%{\footnotesize
%\begin{NoHyper}
%  \tableofcontents[hideallsubsections]
%\end{NoHyper}
%}

\setbeamercolor{enumerate item}{bg=white,fg=black}
\setbeamercolor{item}{bg=white,fg=black}
\setbeamercolor{item projected}{bg=white,fg=black}
\setbeamercolor{enumerate subitem}{fg=red!80!black}
\setbeamertemplate{enumerate items}[default]
\begin{enumerate}
  \item Introduction 
  \item Manipulating Strings
  \item String Processing
  \item Data Processing
  \item Exercises
\end{enumerate}

\end{frame}

\section{Introduction}

\begin{frame}
    \frametitle{}
    \framesubtitle{}
    
    \begin{center}
    {\Huge Part I: Introduction}\\
    {\Large ~}
    \end{center}

\end{frame}

\begin{frame}[fragile]
    \frametitle{Strings}
    \framesubtitle{}

\begin{itemize}[<+->]
  \item A \emph{string} is a sequence of \emph{characters}
  \item ASCII, but more generally may be Unicode
  \item ASCII, CJK characters, emojis, etc.
  \item As of June 2017, Unicode 10.0: 136,755 characters
  \item Support for 1,112,064 characters
  \item We'll stick with ASCII
\end{itemize}

\end{frame}

\begin{frame}[fragile]
    \frametitle{Strings}
    \framesubtitle{}

\begin{itemize}[<+->]
  \item Languages represent strings differently  
  \item In C: a string is an array of \mintinline{c}{char} elements
  \item \alert{Huge caveat}: all strings in C must end (terminate) with
  a null character, \mintinline{text}{\0}
  \item Failure to ensure that all strings are \emph{null-terminated}
  will result in undefined behavior; seg faults, bus errors, etc.
  \item ASCII value of 0, but it is \emph{NOT}:
  \begin{itemize}
    \item \mintinline{text}{'\0'} $\neq$ \mintinline{text}{'0'}
    \item \mintinline{text}{'\0'} $\neq$ \mintinline{text}{'\n'}
    \item \mintinline{text}{'\0'} $\neq$ \mintinline{text}{NULL}
    \item it is the \emph{null terminating character}
  \end{itemize}
\end{itemize}

\end{frame}

\begin{frame}[fragile]
    \frametitle{Strings in C}
    \framesubtitle{}

\begin{itemize}[<+->]
  \item So far: single characters, \mintinline{c}{char}
  \item Character literals denoted with single quotes: \mintinline{c}{'A'}
  \item String literals denoted with double quotes: \mintinline{c}{"Hello World"}
  \item String variables: 
  \begin{itemize}
    \item Static strings: \mintinline{c}{char s[100];}
    \item Static string and initialization: \mintinline{c}{char s[] = "Hello";}
    \item Dynamic strings: \mintinline{c}{char *s;}
    \item Constant Strings: \mintinline{c}{char *s = "Hello";} \onslide<8>{\alert{Don't do this}}
  \end{itemize}
\end{itemize}

\end{frame}

\subsection{Declaration}

\begin{frame}[fragile]
    \frametitle{Static Strings}
    \framesubtitle{}

\begin{itemize}[<+->]
  \item Static string: \mintinline{c}{char s[100];}
  \item Creates a string that can hold \emph{up to 99} characters
  \item Room is needed for the null terminating character
  \item It may hold \emph{shorter} strings, but it may not exceed 99 characters
  \item At declaration, \emph{contents are undefined}
  \item May or may not contain the null terminating character
\end{itemize}
  
\end{frame}

\begin{frame}[fragile]
    \frametitle{Static Strings With Initialization}
    \framesubtitle{}

\begin{itemize}[<+->]
  \item Static string + initialization: \\
    \mintinline{c}{char s[] = "Hello";}
  \item Creates a character array of size 6
  \item Automatically includes the null-terminating character for you
  \item Contents can be changed, strings in C are \emph{mutable}
  \item Static strings are allocated on the stack
\end{itemize}
  
\end{frame}

\begin{frame}[fragile]
    \frametitle{Dynamic Strings}
    \framesubtitle{}

\begin{itemize}[<+->]
  \item Dynamic string:\\
    \mintinline{c}{char *s = (char *) malloc(sizeof(char) * 100);}    
  \item Creates a character array of size 100
  \item May only hold a string that can hold \emph{up to 99} characters
  \item Room is needed for the null terminating character
  \item \emph{Contents are undefined}; may or may not contain the null terminating character
  \item Allocated on the heap
  \item Most of your strings will be of this type
\end{itemize}
  
\end{frame}

\begin{frame}[fragile]
    \frametitle{Constant Strings}
    \framesubtitle{}

\begin{itemize}[<+->]
  \item Constant string declaration: \\
    \mintinline{c}{char *s = "Hello";}   
  \item Dynamic, allocated on the heap, but
  \item Creates a \emph{read-only, immutable} string
  \item Actually uses \mintinline{c}{const char *s = "Hello";} 
  \item But the compiler generally doesn't catch it!
  \item Avoid unless you \emph{really want} a dynamically allocated, immutable string for some reason
\end{itemize}
  
\end{frame}

\subsection{Manipulating Characters}

\begin{frame}[fragile]
    \frametitle{Manipulating Strings by Character}
    \framesubtitle{}

\begin{itemize}[<+->]
  \item Strings are simply character arrays
  \item Each individual element can be \emph{indexed} and 
  \item a value can be assigned to it.
  \item Demonstration
\end{itemize}
  
\end{frame}

\begin{frame}[fragile]
    \frametitle{Manipulating Strings by Character}
    \framesubtitle{}

\begin{minted}[fontsize=\tiny]{c}
char message[] = "hello World!"; //size is 13

message[0] = 'H';
//bad:
message[0] = "H";
printf("message = %s\n", message);

//cut the string short:
message[5] = '\0';
printf("message = %s\n", message);

//restore it: the rest of the contents were unchanged
message[5] = ' ';
printf("message = %s\n", message);

message[11] = '?';
printf("message = %s\n", message);

//bad:
message[12] = '!';
printf("message = %s\n", message);

//really bad:
message = "Goodbye World!";
\end{minted}
  
\end{frame}

\section{String Manipulation}

\begin{frame}
    \frametitle{}
    \framesubtitle{}
    
    \begin{center}
    {\Huge Part II: String Manipulation}\\
    {\Large ~}
    
    %Copy, length, concatenation; 
    % byte-deliminted versions
    \end{center}

\end{frame}

\begin{frame}[fragile]
    \frametitle{Copying Strings}
    \framesubtitle{}

\begin{itemize}[<+->]
  \item You cannot \emph{assign} a string after it has been declared
  \item[~]
\begin{minted}{c}
char s1[] = "hello";
char *s2 = (char *) malloc(sizeof(char) * 6);
\end{minted}
  \item Compiler error: \\
  \mintinline{c}{s1 = "Hello";}
  \item Memory leak: \\
  \mintinline{c}{s2 = "World";}
  \item A string is a character array which is a \emph{memory address}
\end{itemize}
  
\end{frame}

\begin{frame}[fragile]
    \frametitle{Copying Strings}
    \framesubtitle{}

\begin{itemize}[<+->]
  \item You must \emph{copy} the contents of one string into another
  \item String library: \mintinline{c}{string.h} provides a 
  copy function and many others
  \item \mintinline{c}{char * strcpy(char * dest, const char * src);}
  \item Copies the contents of the \emph{source} string into the \emph{destination} string
  \item Assumes \mintinline{c}{src} is properly null-terminated
  \item It is \emph{your} responsibility to ensure that \mintinline{c}{dest}
  is large enough to hold the contents
  \item Demonstration
\end{itemize}
  
\end{frame}

\begin{frame}[fragile]
    \frametitle{Copying Strings}
    \framesubtitle{}

\begin{minted}{c}
char *name = (char *) malloc(sizeof(char) * 10);

strcpy(name, "Chris");
strcpy(name, "Bourke");
//invalid:
strcpy(name, "Chris Bourke");
\end{minted}
  
\end{frame}

\begin{frame}[fragile]
    \frametitle{String Length}
    \framesubtitle{}

\begin{itemize}[<+->]
  \item Essential to know how many characters are stored in a string
  \item \mintinline{c}{size_t strlen(const char *s);}
  \item Result does \emph{not} include the null terminating character!
  \item Demonstration
\end{itemize}
  
\end{frame}

\begin{frame}[fragile]
    \frametitle{String Length}
    \framesubtitle{}

\begin{minted}{c}
char message[] = "Hello World!";
int n = strlen(message);
printf("n = %d\n", n);
message[5] = '\0';
n = strlen(message);
printf("n = %d\n", n);

char * stringCopy(const char *s) {
  char *copy = (char *) malloc(sizeof(char) * (strlen(s) + 1));
  strcpy(copy, s);
  return copy;
}
\end{minted}

\end{frame}

\begin{frame}[fragile]
    \frametitle{String Concatenation}
    \framesubtitle{}

\begin{itemize}[<+->]
  \item Copying overwrites a string's contents
  \item Alternative: append or \emph{concatenation} the contents of one string onto the end of another
  \item \mintinline{c}{char *strcat(char *dest, const char *src);}
  \item Assumes both are null-terminated
  \item Assumes \mintinline{c}{dest} is large enough to hold both
  \item Demonstration
\end{itemize}
  
\end{frame}

\begin{frame}[fragile]
    \frametitle{String Concatenation}
    \framesubtitle{}

\begin{minted}{c}
char *firstName = (char *) malloc((5+1) * sizeof(char));
char *lastName = (char *) malloc((6+1) * sizeof(char));
char *str = (char *) malloc(101) * sizeof(char));

strcpy(firstName, "Chris");
strcpy(lastName, "Bourke");

strcpy(str, lastName);
strcat(str, ", ");
strcat(str, firstName);
//str contains "Bourke, Chris"
\end{minted}

\end{frame}

\begin{frame}[fragile]
    \frametitle{Byte-Limited Versions}
    \framesubtitle{}

\begin{itemize}[<+->]
  \item String library provides \emph{byte-limited} versions of
  both copy and concatenation functions
  \item \mintinline{c}{char *strncat(char *dest, const char *src, size_t n);}
  \item \mintinline{c}{char *strncpy(char *dest, const char *src, size_t n);}
  \item Only copies \emph{at most} first \mintinline{c}{n} bytes/characters
  \item Stops early if it sees \mintinline{text}{\0}
  \item Includes \mintinline{text}{\0} \emph{only if} it is within the first
  \mintinline{c}{n} bytes!
  \item Demonstration
\end{itemize}
  
\end{frame}

\begin{frame}[fragile]
    \frametitle{Byte-Limited Versions}
    \framesubtitle{}

\begin{minted}{c}
char fullName[] = "Christopher";
char *nickName[] = (char *) malloc(6 * sizeof(char));

strncpy(nickName, fullName, 5);
//don't forget:
firstName[5] = '\0';
\end{minted}

\end{frame}

\section{String Processing}

\begin{frame}
    \frametitle{}
    \framesubtitle{}
    
    \begin{center}
    {\Huge Part III: String Processing}\\
    {\Large ~}
    
    %Others:
    % ctype
    % string formatting
    % string comparisons
    % substring using &str[i]
    % arrays of strings
    \end{center}

\end{frame}

\subsection{Iterations}

\begin{frame}[fragile]
  \frametitle{Iterating Over Strings}
  \framesubtitle{}

\begin{itemize}[<+->]
  \item Common to iterate over a string character-by-character
  \item Straightforward solution: for-loop using \mintinline{c}{strlen()}
  \item Generally better ways
  \item Demonstration
\end{itemize}

\end{frame}

\begin{frame}[fragile]
  \frametitle{Iterating Over Strings}
  \framesubtitle{}

\begin{minted}[fontsize=\tiny]{c}
//straightforward code:
for(int i=0; i<strlen(s); i++) { 
  printf("%c\n", s[i]);
}

//strlen works like:
int i = 0;
while(s[i] != '\0') {
  i++;
}
//the value of i equals strlen(s)

//optimized:
int n = strlen(s);
for(int i=0; i<n; i++) { 
  printf("%c\n", s[i]);
}

//even better:
for(int i=0; s[i] != '\0'; i++) {
  printf("%c\n", s[i]);
}
\end{minted}

\end{frame}

\subsection{ctype library}

\begin{frame}[fragile]
  \frametitle{ctype library}
  \framesubtitle{}
  
\begin{itemize}[<+->]
  \item May want to process or manipulate individual characters
  \item The \mintinline{c}{ctype.h} library provides many useful character functions
  \item Functions use ASCII \mintinline{c}{int} values
  \item Automatically type casted 
\end{itemize}

    
\end{frame}


\begin{frame}[fragile]
  \frametitle{ctype library}
  \framesubtitle{}
  
\begin{itemize}[<+->]  \item \mintinline{c}{int isdigit(int c)} -- returns true if \mintinline{c}{c} is a digit character, \mintinline{c}{0} thru \mintinline{c}{9}
  \item \mintinline{c}{int islower(int c)} -- returns true if \mintinline{c}{c} is a lowercase letter character, \mintinline{c}{a} thru \mintinline{c}{z}
  \item \mintinline{c}{int isupper(int c)} -- returns true if \mintinline{c}{c} is an uppercase letter character, \mintinline{c}{A} thru \mintinline{c}{Z}
  \item \mintinline{c}{int isspace(int c)} -- returns true if \mintinline{c}{c} is a whitespace character: space, tab, newline, etc.
  \item \mintinline{c}{int tolower(int c)}, \mintinline{c}{int toupper(int c)} -- return the ASCII text value of the lowercase/uppercase version of \mintinline{c}{c}
  \item Demonstration: write a code snippet to count the number of spaces and the total number of whitespace characters.
\end{itemize}

\end{frame}


\begin{frame}[fragile]
  \frametitle{ctype library}
  \framesubtitle{}

\begin{minted}{c}
char str[] = "Hello how \n\t are you today? \n\n";
int numSpaces = 0;
int numWhiteSpaces = 0;

for(int i=0; s[i] != '\0'; i++) {
  if(isspace(s[i])) {
    numWhiteSpaces++;
  }
  if(s[i] == ' ') {
    numSpaces++;
  }
}
printf("number of spaces: %d\n", numSpaces);
printf("total whitespace: %d\n", numWhiteSpaces);
\end{minted}
    
\end{frame}

\subsection{String Comparisons}

\begin{frame}[fragile]
  \frametitle{String Comparisons}
  \framesubtitle{}
  
\begin{itemize}[<+->]
  \item Often need to compare \emph{entire strings} for equality
  \item You \emph{cannot} use the equality operator!
  \item \mintinline{c}{s1 == s2} compares memory addresses!
  \item Need to use:\\
    \mintinline{c}{int strcmp(const char *str1, const char *str2)} 
  \item Returns:
  \begin{itemize}
    \item $< 0$ if \mintinline{c}{str1} comes before \mintinline{c}{str2}
    \item $0$ if contents are equal
    \item $> 0$ if \mintinline{c}{str1} comes after \mintinline{c}{str2}
  \end{itemize}
  \item Order is determined by ASCII text values
  \item Demonstration
\end{itemize}
    
\end{frame}

\begin{frame}[fragile]
  \frametitle{String Comparisons}
  \framesubtitle{}

\begin{minted}{c}
int result;

result = strcmp("apple", "apple"); //0
result = strcmp("apple", "apples"); //negative
result = strcmp("apples", "apple"); //positive
result = strcmp("Apple", "apple"); //negative
result = strcmp("apples", "oranges"); //negative

result = strcmp("100", "99"); //negative!

result = strncmp("apple", "apples", 5); //zero

result = strcasecmp("ApPlE", "apple"); //zero
\end{minted}

\end{frame}

\subsection{Substrings}


\begin{frame}[fragile]
  \frametitle{Substrings}
  \framesubtitle{}
  
\begin{itemize}[<+->]
  \item It is possible to reference a \emph{substring} of a string
  \item Reference a part of the string starting at a particular index
  \item Demonstration
\end{itemize}
    
\end{frame}

\begin{frame}[fragile]
  \frametitle{Substrings}
  \framesubtitle{}
  
\begin{minted}{c}
char name[] = "Margaret Hamilton";
char *lastName = &name[9];
printf("Greetings, Ms. %s\n");
\end{minted}
    
\end{frame}

\subsection{Formatting Strings}


\begin{frame}[fragile]
  \frametitle{Formatting Strings}
  \framesubtitle{}

\begin{itemize}[<+->]
  \item Already familiar with \mintinline{c}{%s} placeholder to print a string
  to the standard output
  \item \mintinline{c}{atoi} and \mintinline{c}{atof} convert strings to numbers
  \item Possible to convert numbers to strings
  \item ``Print'' to a string instead of the standard output
  \item \mintinline{c}{sprintf()}: print to a string
  \item Demonstration
\end{itemize}
    
\end{frame}

\begin{frame}[fragile]
  \frametitle{Formatting Strings}
  \framesubtitle{}
  
\begin{minted}{c}
char s[100]; //buffer that is "big enough"
char state[] = "Nebraska";
int numCounties = 93;
double population = 1.92;
sprintf(s, "%s has %d counties and a population of %.2f million.\n", state,numCounties, population);
\end{minted}
    
\end{frame}

\subsection{Arrays of Strings}

\begin{frame}[fragile]
  \frametitle{Arrays of Strings}
  \framesubtitle{}
  
\begin{itemize}[<+->]
  \item Arrays of strings are simply 2-D arrays of \mintinline{c}{char} elements
  \item Each ``row'' is a string that must be null-terminated
  \item Each row/string need not be the same size
  \item Easy extension of 2-D arrays: \mintinline{c}{char **arrayOfStrings}
  \item Demonstration
\end{itemize}
    
\end{frame}

\begin{frame}[fragile]
  \frametitle{Arrays of Strings}
  \framesubtitle{}
  
\begin{minted}{c}
char **names = (char **) malloc(sizeof(char*) * 5);
names[0] = stringCopy("Margaret Hamilton");
names[1] = stringCopy("Grace Hopper");
names[2] = stringCopy("Alan Turing");
names[3] = stringCopy("Ada Lovelace");
names[4] = stringCopy("Dennis Ritchie");
for(int i=0; i<5; i++) {
  printf("Famous Computer Scientist: %s\n", names[i]);
}
\end{minted}
    
\end{frame}

\section{Data Processing}

\begin{frame}
    \frametitle{}
    \framesubtitle{}
    
    \begin{center}
    {\Huge Part IV: Data Processing}\\
    {\Large ~}
    
    %Tokenization
    \end{center}

\end{frame}

\begin{frame}[fragile]
    \frametitle{String Tokenization}
    \framesubtitle{}

\begin{itemize}[<+->]
  \item Strings may contain formatted data: CSV, TSV
  \item \emph{Tokenization} is the process of splitting a string along some 
  \emph{delimiter} and processing each \emph{token} separately
  \item Example: \mintinline{c}{"Hedy,Lamarr,UNL,Avery Hall,Lincoln,NE"}
  \item Tokens: \\
  \mintinline{c}{"Hedy"} \mintinline{c}{"Lamarr"} \mintinline{c}{"UNL"}
  \mintinline{c}{"Avery"} \mintinline{c}{"Hall"} \mintinline{c}{"Lincoln"}
  \mintinline{c}{"NE"}
  \item Generally ignore the delimiter
\end{itemize}

\end{frame}

\begin{frame}[fragile]
    \frametitle{String Tokenization}
    \framesubtitle{}
    
\begin{itemize}[<+->]
  \item \mintinline{c}{char * strtok(char *str, const char *delim);} 
  \item Tokenizes \mintinline{c}{str} along instances of \mintinline{c}{delim}
  \item Usage:
  \begin{itemize}
    \item First time you call it: pass the string to be tokenized
    \item Subsequent calls: pass \mintinline{c}{NULL} to continue with the same string
    \item Returns a pointer to the next token 
    \item It \emph{modifies your string!}
  \end{itemize}
  \item Demonstration
\end{itemize}

\end{frame}

\begin{frame}[fragile]
    \frametitle{String Tokenization}
    \framesubtitle{}
    
\begin{minted}{c}
  char str[] = "Hedy,Lamarr,UNL,Avery Hall,Lincoln,NE";
  char *token = NULL;
  token = strtok(str, ",");
  while(token != NULL) {
    printf("token = %s\n", token);
    token = strtok(NULL, ",");
  }
\end{minted}

\end{frame}

\section{Exercises}

\begin{frame}
    \frametitle{}
    \framesubtitle{}
    
    \begin{center}
    {\Huge Part V: Exercises}\\
    {\Large ~}
    
    \end{center}

\end{frame}


\begin{frame}
    \frametitle{}
    \framesubtitle{}
    
\begin{itemize}
  %\item Write a string function to create a deep copy of a string
  \item Write a string function to change a string's characters to uppercase letters
  \item Write a string function that returns a new copy of a string with all characters converted to uppercase
  %\item remove char
  \item Write a string function to ``double space'' a paragraph %: every endline character instance should be doubled.
  \item Write a ``split''-style function: it takes a string and a delimiter 
  and returns an array of strings of the tokens.  Ensure no memory leaks!
\end{itemize}

\end{frame}

\end{document} 
