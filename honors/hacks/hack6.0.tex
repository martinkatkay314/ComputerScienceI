\documentclass[12pt]{scrartcl}


\usepackage{epsfig,amssymb}

\usepackage{xcolor}
\usepackage{graphicx}
\usepackage{epstopdf}
\usepackage{multirow}
\usepackage{float}

\definecolor{darkred}{rgb}{0.5,0,0}
\definecolor{darkgreen}{rgb}{0,0.5,0}
\usepackage[pdfusetitle]{hyperref}
\hypersetup{
  letterpaper,
  colorlinks,
  linkcolor=red,
  citecolor=darkgreen,
  menucolor=darkred,
  urlcolor=blue,
  pdfpagemode=none,
}

\usepackage{fullpage}
\usepackage{tikz}
\pagestyle{empty} %
%obsolete: \usepackage{subfigure}
%use: 
\usepackage{subcaption}

\definecolor{MyDarkBlue}{rgb}{0,0.08,0.45}
\definecolor{MyDarkRed}{rgb}{0.45,0.08,0}
\definecolor{MyDarkGreen}{rgb}{0.08,0.45,0.08}

\definecolor{mintedBackground}{rgb}{0.95,0.95,0.95}
\definecolor{mintedInlineBackground}{rgb}{.90,.90,1}

\usepackage[newfloat=true]{minted}

\setminted{mathescape,
           linenos,
           autogobble,
           frame=none,
           framesep=2mm,
           framerule=0.4pt,
           %label=foo,
           xleftmargin=2em,
           xrightmargin=0em,
           %startinline=true,  %PHP only, allow it to omit the PHP Tags *** with this option, variables using dollar sign in comments are treated as latex math
           numbersep=10pt, %gap between line numbers and start of line
           style=default} %syntax highlighting style, default is "default"

\setmintedinline{bgcolor={mintedBackground}}
%doesn't work with the above workaround:
\setminted{bgcolor={mintedBackground}}
\setminted[text]{bgcolor={mintedBackground},linenos=false,autogobble,xleftmargin=1em}
%\setminted[php]{bgcolor=mintedBackgroundPHP} %startinline=True}
\SetupFloatingEnvironment{listing}{name=Code Sample}
\SetupFloatingEnvironment{listing}{listname=List of Code Samples}

\setlength{\parindent}{0pt} %
\setlength{\parskip}{.25cm}
\newcommand{\comment}[1]{}

\usepackage{amsmath}
\usepackage{algorithm2e}
\SetKwInOut{Input}{input}
\SetKwInOut{Output}{output}
%NOTE: you can embed algorithms in solutions, but they cannot be floating objects; use [H] to make them non-floats

\usepackage{lastpage}

%\usepackage{titling}
\usepackage{fancyhdr}
\renewcommand*{\titlepagestyle}{fancy}
\pagestyle{fancy}
%\fancyhf{}
%\rhead{Computer Science I}
%\lhead{Guides and tutorials}
\renewcommand{\headrulewidth}{0.0pt}
\renewcommand{\footrulewidth}{0.4pt}
\lfoot{\Title\ -- Computer Science I}
\cfoot{~}
\rfoot{\thepage\ / \pageref*{LastPage}}


\makeatletter
\title{Hack 6.0}\let\Title\@title
\subtitle{Computer Science I -- Java\\
{\small
\vskip1cm
Department of Computer Science \& Engineering \\
University of Nebraska--Lincoln}
\vskip-1cm}
%\author{Dr.\ Chris Bourke}
\date{~}
\makeatother

\begin{document}

\maketitle

\hrule

\section*{Introduction}

Hack session activities are small weekly programming assignments intended
to get you started on full programming assignments.  Collaboration is allowed
and, in fact, \emph{highly encouraged}.  You may start on the activity before
your hack session, but during the hack session you must either be actively 
working on this activity or \emph{helping others} work on the activity.
You are graded using the same rubric as assignments so documentation, style, 
design and correctness are all important.

%\subsection*{Rubric}
%\begin{table}[H]
%\begin{tabular}{ll}
%Category       & Point Value \\
%Style          & 2           \\
%Documentation  & 2           \\
%Design         & 5           \\
%Correctness    & 16          \\
%\textbf{Total} & \textbf{25}
%\end{tabular}
%\end{table}



\section*{Problem Statement}

In this hack you'll get some more practice writing methods
, error handling and enumerated types.  There are several different
ways to model colors including RGB and CMYK.  RGB is generally used in displays
and models a color with three values in the range $[0, 255]$ corresponding to 
the red, green and blue ``contribution'' to the color.  For example, the
triple $(255, 255, 0)$ corresponds to a full red and green (additive) value
which results in yellow.  CMYK or Cyan-Magenta-Yellow-Black is a model used
in printing where four colors of ink are combined to make various colors.
In this system, the four values are on the scale $[0, 1]$.  Write 
functions to convert between these models.

\begin{enumerate}
\item Write a function to convert from an RGB color model to CMYK.  To 
convert to CMYK, you first need to scale each integer value to the range 
$[0, 1]$ by simply computing
	$$r' = \frac{r}{255}, \quad g' = \frac{g}{255}, \quad b' = \frac{b}{255}$$
	and then using the following formulas:
\begin{align*}
k & = 1-\max\{r', g', b'\} \\
c & = \frac{(1-r'-k)}{(1-k)} \\
m & = \frac{(1-g'-k)}{(1-k)} \\
y & = \frac{(1-b'-k)}{(1-k)} \\
\end{align*}
Your method should have the following signature:

\mintinline{java}{public static CMYK rgbToCMYK(RGB color)}

Note that one edge case is black, when $(r,g,b) = (0,0,0)$ which would lead to a 
division by zero in the formulas.  The equivalent CMYK values are $(0,0,0,1)$.


\item Write a function to convert from CMYK to RGB using the following formulas.
\begin{align*}
r & = 255 \cdot (1 - c) \cdot (1-k) \\
g & = 255 \cdot (1 - m) \cdot (1-k) \\
b & = 255 \cdot (1 - y) \cdot (1-k) \\
\end{align*}
Results should be rounded.  Your method should have the following signature:

\mintinline{java}{public static RGB cmykToRGB(CMYK color)}

\end{enumerate}

The \mintinline{java}{RGB} and \mintinline{java}{CMYK} classes have 
been provided for you.  See the \mintinline{java}{main} method of 
each class for examples on how to create and use \emph{instances} 
of each class.

Place both methods in a source file named \mintinline{text}{ColorUtils.java}
in the package \mintinline{java}{unl.cse}.  For both methods, identify any and 
all error conditions and throw an \mintinline{java}{IllegalArgumentException} 
with an appropriate error message.


\section*{Instructions}

\begin{itemize}
  \item You are encouraged to collaborate any number of students 
  before, during, and after your scheduled hack session.  
  \item Design at least 3 test cases for each function
  \emph{before} you begin
  designing or implementing your program.  Test cases are 
  input-output pairs that are known to be correct using means
  other than your program.
  \item You may (in fact are encouraged) to define any additional
  ``helper'' functions that you find useful.
  \item Include the name(s) of everyone who worked together on
  this activity in your source file's header.

  \item A testing file, \mintinline{text}{ColorUtilsTests.java} has been 
  provided that uses JUnit (\url{https://junit.org/junit5/}), a unit testing 
  framework for Java.  We have already written several test cases 
  for you.  Using these examples, implement your test cases using JUnit
  for your two functions.  You should add at least 3 test methods.

  The starter file should be sufficient to demonstrate how to use
  JUnit, but the full documentation can be found here: 
  \url{https://junit.org/junit5/docs/current/api/}.
  A \mintinline{text}{readme.me} file has also been provided describing
  how to add JUnit to your Eclipse project.   
  
  \item Turn in all of your files via webhandin, making sure that 
  it runs and executes correctly in the webgrader.  Each individual 
  student will need to hand in their own copy and will receive 
  their own individual grade.
\end{itemize}  


\end{document}
